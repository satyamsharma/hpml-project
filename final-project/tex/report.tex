\documentclass[conference]{IEEEtran}
% \IEEEoverridecommandlockouts
% The preceding line is only needed to identify funding in the first footnote. If that is unneeded, please comment it out.
\usepackage{cite}
\usepackage{amsmath,amssymb,amsfonts}
\usepackage{algorithmic}
\usepackage{graphicx}
\usepackage{textcomp}
\usepackage{xcolor}
\usepackage{booktabs}
\def\BibTeX{{\rm B\kern-.05em{\sc i\kern-.025em b}\kern-.08em
    T\kern-.1667em\lower.7ex\hbox{E}\kern-.125emX}}
\begin{document}

\title{Comparing Hyperdimensional Computing to Deep Learning for Natural Language Processing Tasks}

\author{\IEEEauthorblockN{Todd Morrill}
\IEEEauthorblockA{\textit{Computer Science Department} \\
\textit{Columbia University}\\
tm3229@columbia.edu}
\and
\IEEEauthorblockN{Satyam Sharma}
\IEEEauthorblockA{\textit{Computer Science Department} \\
\textit{Columbia University}\\
ss6522@columbia.edu}
}

\maketitle

\begin{abstract}
    In this project, we will compare the performance of deep learning models (e.g. Transformers, Convolutional Neural Networks (CNN), Long Short-Term Memory (LSTM) models) to HDC models on a variety of NLP tasks using a range of metrics and evaluate their relative strengths and weaknesses.
\end{abstract}

\begin{IEEEkeywords}
hyperdimensional computing, HDC, deep learning, natural language processing, NLP
\end{IEEEkeywords}

\section{Introduction}
\cite{BEIR}

\section{Results}

% include table
\begin{table}[htbp]
\caption{HDC accuracy scores by dataset size.}
\begin{center}
    \begin{tabular}{ccc}
\toprule
 Examples &  Dataset Pct. &  Accuracy \\
\midrule
       21 &        0.0001 &    0.2682 \\
      210 &        0.0010 &    0.8375 \\
     2100 &        0.0100 &    0.9593 \\
     4200 &        0.0200 &    0.9659 \\
    10501 &        0.0500 &    0.9700 \\
\bottomrule
\end{tabular}

\end{center}
\label{tab:hdc_acc}
\end{table}



\bibliographystyle{IEEEtran}
\bibliography{IEEEabrv, bib.bib}
\end{document}
