\documentclass[conference]{IEEEtran}
% \IEEEoverridecommandlockouts
% The preceding line is only needed to identify funding in the first footnote. If that is unneeded, please comment it out.
\usepackage{cite}
\usepackage{amsmath,amssymb,amsfonts}
\usepackage{algorithmic}
\usepackage{graphicx}
\usepackage{textcomp}
\usepackage{xcolor}
\usepackage{booktabs}
\def\BibTeX{{\rm B\kern-.05em{\sc i\kern-.025em b}\kern-.08em
    T\kern-.1667em\lower.7ex\hbox{E}\kern-.125emX}}
\begin{document}

\title{Comparing Hyperdimensional Computing to Deep Learning for Natural Language Processing Tasks}

\author{\IEEEauthorblockN{Todd Morrill}
\IEEEauthorblockA{\textit{Computer Science Department} \\
\textit{Columbia University}\\
tm3229@columbia.edu}
\and
\IEEEauthorblockN{Satyam Sharma}
\IEEEauthorblockA{\textit{Computer Science Department} \\
\textit{Columbia University}\\
ss6522@columbia.edu}
}

\maketitle

\begin{abstract}
    In this project, we will compare the performance of deep learning models (e.g. Transformers, Convolutional Neural Networks (CNN), Long Short-Term Memory (LSTM) models) to HDC models on a variety of NLP tasks using a range of metrics and evaluate their relative strengths and weaknesses.
\end{abstract}

\begin{IEEEkeywords}
hyperdimensional computing, HDC, deep learning, natural language processing, NLP
\end{IEEEkeywords}

\section{Introduction}
% Introduce the problem you are solving and motivation behind it
Deep learning is currently the dominant approach in natural language processing (NLP). In particular, the Transformer \cite{Transformer} architecture has been shown to be effective for a wide range of NLP tasks, including language modeling (i.e. next word prediction), document retrieval, and document classification \cite{SuperGLUE}. However, deep learning models require a large amount of training data and are often memory and energy intensive, which limit their usability on low-resource devices (e.g. smartphones). Hyperdimensional computing (HDC), on the other hand, is a neuro-inspired approach to machine learning that is memory and energy efficient and may require far less training data to achieve suitable levels of accuracy \cite{HDC}. In short, HDC typically represents data as random high dimensional vectors (e.g. a word may be represented in $\{-1, 1\}^{10,000}$). HDC uses a variety of elementwise operations to operate on this data. In particular, addition is coordinatewise majority, mulitiplication is coordinatewise XOR, permutation is a rotation of coordinates (i.e. shift to the right), and comparison can be done using a hamming distance or cosine distance. These operations allow HDC to perform next word prediction, document retrieval, and classification, among other tasks.

In this project, we will compare the performance of deep learning models (e.g. Transformers, Convolutional Neural Networks (CNN), Long Short-Term Memory (LSTM) models) to HDC models on a variety of NLP tasks using a range of metrics and evaluate their relative strengths and weaknesses.

\section{Models and Data Set(s) Description}
We aim to compare the performance of deep learning and HDC on the following NLP tasks: 
\begin{enumerate}
    \item missing word prediction - predicting the masked word in a given sentence
    \item document retrieval - querying semantically similar content across several documents
    \item language identification - detecting the language of a given sentence.
\end{enumerate}

To determine relative strengths and weaknesses, we will be capturing the following metrics:
\begin{enumerate}
    \item accuracy relative to different dataset sizes
    \item training and inference time
    \item energy consumption as measured by FLOPs \cite{desislavov2021compute}
    \item robustness against input data corruption
\end{enumerate}

\section{Training and Profiling Methodology}
We will primarily use Python for our implementation and use PyTorch and Hugging Face \cite{HF} to implement the deep learning models and TorchHD \cite{torchhd} to implement the HDC models. We intend to implement and run our experiments on a combination of laptops, Habanero, Google Colab, and a personal deep learning server. We plan to train and evaluate on the following corpora:
\begin{enumerate}
    \item missing word prediction - Wikipedia \cite{wikipedia}
    \item document retrieval - the BEIR benchmark \cite{BEIR}
    \item language identification - the Wortschatz Corpora \cite{quasthoff-etal-2006-corpus}.
\end{enumerate}

\section{Performance Tuning Methodology}

\section{Experimental Results}

\section{Conclusion}

\section{Appendix}
% include table
\begin{table}[htbp]
\caption{HDC accuracy scores by dataset size.}
\begin{center}
    \begin{tabular}{ccc}
\toprule
 Examples &  Dataset Pct. &  Accuracy \\
\midrule
       21 &        0.0001 &    0.2682 \\
      210 &        0.0010 &    0.8375 \\
     2100 &        0.0100 &    0.9593 \\
     4200 &        0.0200 &    0.9659 \\
    10501 &        0.0500 &    0.9700 \\
\bottomrule
\end{tabular}

\end{center}
\label{tab:hdc_acc}
\end{table}

\begin{table}[htbp]
    \caption{Deep learning accuracy scores by dataset size.}
    \begin{center}
        \begin{tabular}{ccc}
\toprule
Examples & Dataset Pct. & Accuracy \\
\midrule
21 & 0.0001 & 0.0476 \\
210 & 0.0010 & 0.0541 \\
2100 & 0.0100 & 0.7886 \\
4200 & 0.0200 & 0.8629 \\
10501 & 0.0500 & 0.9513 \\
21003 & 0.1000 & 0.9652 \\
42006 & 0.2000 & 0.9793 \\
105016 & 0.5000 & 0.9855 \\
210032 & 1.0000 & 0.9898 \\
\bottomrule
\end{tabular}

    \end{center}
    \label{tab:dl_acc}
\end{table}

\begin{table}[htbp]
    \caption{Deep learning speed analysis.}
    \begin{center}
        \begin{tabular}{ccc}
\toprule
Model & Training-Time & Testing-Time \\
\midrule
HDC & 101.395437 & 10.215967 \\
HDC-Optimized & 44.142246 & 4.526111 \\
distilbert-base-uncased & 1559.166500 & 88.025828 \\
\bottomrule
\end{tabular}

    \end{center}
    \label{tab:dl_speed}
\end{table}

\begin{table}[htbp]
    \caption{Deep learning FLOP analysis.}
    \begin{center}
        \begin{tabular}{ccc}
\toprule
Model & Parameters & FLOPs \\
\midrule
HDC & 0 & 0 \\
distilbert-base-uncased & 43135509 & 936371712 \\
\bottomrule
\end{tabular}

    \end{center}
    \label{tab:dl_flop}
\end{table}

\bibliographystyle{IEEEtran}
\bibliography{IEEEabrv, bib.bib}
\end{document}
